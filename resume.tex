\documentclass[]{resume}

\usepackage{fancyhdr}

\pagestyle{fancy}
\fancyhf{}
\begin{document}

\namesection{Meesum}{A. Qazalbash}{\thinspace CS Senior at Habib University}{\thinspace Mathematics \& Competitve Programming Enthusiast}

\contactline{\href{mailto:meesumqazalbash@gmail.com}{meesumqazalbash@gmail.com}}{\href{https://www.github.com/Qazalbash}{Qazalbash}}{\href{https://www.linkedin.com/in/meesumaliqazalbash//}{li-meesumaliqazalbash}}{\href{tel:+923351120129}{+923351120129}}

%%%%%%%%%%%%%%%%%%%%%%%%%%%%%%%%%%%%%%
%
%     COLUMN ONE
%
%%%%%%%%%%%%%%%%%%%%%%%%%%%%%%%%%%%%%%

\begin{minipage}[t]{0.5\textwidth}

    %%%%%%%%%%%%%%%%%%%%%%%%%%%%%%%%%%%%%%
    %     EXPERIENCE
    %%%%%%%%%%%%%%%%%%%%%%%+%%%%%%%%%%%%%%%

    \section{Experience}
    \runsubsection{Habib University}

    \descript{Teacher Assistant}
    \location{Aug 2021 - Present | Karachi, Pak}
    \vspace{\topsep} % Hacky fix for awkward extra vertical space
    \begin{tightemize}
        \sectionsep
        \item Teaching Assistant for courses Programming Fundamentals, Data Structures and Algorithms, Object Oriented Programming, Calculus I and Engineering Mathematics.
        \item My responsibilities include checking assignments, and assisting in labs.
    \end{tightemize}
    \descript{Cofounder \& Lead of Habib Maths Club}
    \location{Aug 2021 - Present | Karachi, Pak}
    % \vspace{\topsep} % Hacky fix for awkward extra vertical space
    \begin{tightemize}
        \sectionsep
        \item I along with my team arranged intependent study groups for Abstract Algebra, Number Theory and Topology.
        \item We also arranged a competition for secondary school and high school students named as Mathema.
    \end{tightemize}

    \descript{Workshop Instructor}
    \location{Dec 2022 - Jan 2023 | Karachi, Pak}
    \begin{tightemize}
        \sectionsep
        \item Conducted the winter workshop of Data Structure \& Algorithm (Python3) and Object-Oriented Programming (C++) for 40 students.
    \end{tightemize}

    \runsubsection{Summer Tehqiq Research Program}

    \descript{Undergraduate Researcher}
    \location{Jun 2022 - Aug 2022 | Karachi, Pak}
    \begin{tightemize}
        \sectionsep
        \item Worked on a research project to emulate the working of graphical pipeline on a CPU.
        \item The project was aimed to promote the hands on experience of students in the field of Computer Graphics. Later the same project worked as the prototype for the pipeline students were supposed to implement in their course project.
    \end{tightemize}

    %%%%%%%%%%%%%%%%%%%%%%%%%%%%%%%%%%%%%%
    %     EDUCATION
    %%%%%%%%%%%%%%%%%%%%%%%%%%%%%%%%%%%%%%

    \section{Education}
    \runsubsection{Habib University} \descript{BSc. in Computer Science \& Mathematics}
    \location{Aug 2020 - May 2024 | Karachi, Pak}
    \begin{tightemize}
        \sectionsep
        \item List of all courses taken is available on \href{https://www.linkedin.com/in/meesumaliqazalbash/details/courses/}{LinkedIn}.
        \item CGPA: 3.8/4.0
        \item Dean's List: Fall 2022.
        \item Cofounder \& Lead of Habib Maths Club.
    \end{tightemize}

    \sectionsep

    \runsubsection{Adamjee Government Science College} \descript{Intermediate Pre-Engineering}
    \location{Aug 2018 - Jun 2020 | Karachi, Pak}
    \begin{tightemize}
        \item Secured A1 Grade (Overall 91\%) in HSSC.
        \item 7th position citywide.
    \end{tightemize}

    % \subsection{Certificate Courses}
    % DataCamp:
    % \begin{tightemize}
    %     \item Intermediate Python
    %     \item Introduction to Data Science
    %     \item Data Manipulation with Pandas
    %     \item Data Joining with Python
    %     \item Data Cleaning with Python
    %     \item Visualization with Matplotlib
    % \end{tightemize}
    % Udemy:
    % \begin{tightemize}
    %     \item The Complete 2021 Flutter Development Bootcamp with Dart
    % \end{tightemize}
    % GreatLearning:
    % \begin{tightemize}
    %     \item Android Application Development (Flutter)
    % \end{tightemize}
    % KhanAcademy:
    % \begin{tightemize}
    %     \item Intro to HTML/CSS: Making webpages
    % \end{tightemize}
    % Coursera:
    % \begin{tightemize}
    %     \item Object-Oriented Programming in Python
    %     % \item FPGA Design for Embedded Systems
    %     \item Hardware Description Languages for FPGA Design
    % \end{tightemize}


    %%%%%%%%%%%%%%%%%%%%%%%%%%%%%%%%%%%%%%
    %     SKILLS
    %%%%%%%%%%%%%%%%%%%%%%%%%%%%%%%%%%%%%%

    \section{Awards}
    \sectionsep
    \begin{tightemize}
        \item Dean's List for Fall 2022.
        \item Merit based 100\% scholarship under the name of Talent Outreach Promotion \& Support Program by Habib University.
    \end{tightemize}

\end{minipage}
\hfill
\begin{minipage}[t]{0.5\textwidth}


    \section{Skills}

    \subsection{Tools \& Softwares}
    Linux (Proficient with Ubuntu, Archlinux),
    Git Versioning Tool (Proficient),
    Jupyter Notebook (Proficient),
    \LaTeX (Proficient),
    Markdown (Proficient),
    Plotly (Intermediate),
    Dash (Intermediate), Bash (Basic),
    HTML (Basic), CSS (Basic)\\

    \subsection{Programming Languages}
    Python (Expert), C/C++ (Expert), C-OpenCL (Intermediate), CUDA-C/C++ (Intermediate), JavaScript (Basic), Julia (Basic), Haskell (Basic)

    %%%%%%%%%%%%%%%%%%%%%%%%%%%%%%%%%%%%%%
    %
    %     COLUMN TWO
    %
    %%%%%%%%%%%%%%%%%%%%%%%%%%%%%%%%%%%%%%

    % % %%%%%%%%%%%%%%%%%%%%%%%%%%%%%%%%%%%%%%
    % % %     COMPETITIONS
    % % %%%%%%%%%%%%%%%%%%%%%%%%%%%%%%%%%%%%%%
    \section{Content Writing}
    I have been writing social and tech related blogs for past two years. You can find my work on \href{https://medium.com/@mesumali26-ma}{Medium}.

    \section{Competitive Programming}
    I have been doing Competitive Programming for more than two years.\\
    \subsection{\href{https://leetcode.com/maq2628/}{\textbf{LeetCode - Profile (maq2628)}}}
    \textbf{ICPC Asia Regionals 2023}\\
    My team qualified for the Regional round, we were unable to attend the event due to fiscal constraints.\\
    \textbf{ICPC Asia Regionals 2022}\\
    Secured 19th position nationwide, solved 4/7 questions.

    %%%%%%%%%%%%%%%%%%%%%%%%%%%%%%%%%%%%%%
    %              Portfolio              %
    %%%%%%%%%%%%%%%%%%%%%%%%%%%%%%%%%%%%%%

    \section{Portfolio}
    %  (Hyperlinks attached otherwise these projects are present at my GitHub profile)\\

    \subsection{\href{https://github.com/Qazalbash/GeneTime}{\textbf{GeneTime}}}
    A genetic algorithm that optimizes the time table for a university provided some soft and hard constraints. It is written in \textbf{Python3}.

    \subsection{\href{https://github.com/Qazalbash/Raytracer}{\textbf{Raytracer}}}% - OOP
    A Raytracer written in \textbf{C++} which can render 3D scenes with spheres, triangles, and planes.

    \subsection{\href{https://github.com/Qazalbash/Emulated-Graphical-Pipeline}{
            \textbf{Emulated Graphical Pipeline}}}% - Computer Graphics
    An Emulated Graphical Pipeline written in \textbf{Python3} which can render basic scenes with spheres, triangles, and planes.

    \subsection{\href{https://github.com/Qazalbash/Processor-on-Verilog}{\textbf{Processor on Verilog}}}
    A Processor written in \textbf{Verilog} which can perform basic arithmetic and logical operations. They are of two types: Single Cycle and Pipeline Processor with hazard detection and forwarding.

    \subsection{\href{https://github.com/Qazalbash/Checkers-with-AI}{\textbf{Checkers with AI}}}% - DataStructures II
    A Checkers game developed in \textbf{Python3} with an AI that uses greedy algorithm to play.

    \subsection{\href{https://github.com/Qazalbash/Flight-Simulator-in-WebGL}{\textbf{Flight Simulator}}}
    A Flight Simulator written in \textbf{WebGL} which can render 3D scenes with planes, mountains, and clouds.

    \subsection{\href{https://github.com/Qazalbash/Wordle-Assistant}{\textbf{Wordle Assistant}}}% - DataStructures I
    A Wordle Assistant written in \textbf{C++} that helps in solving Wordle puzzles by using inferencing and backtracking.

    \subsection{\href{https://github.com/Qazalbash/Dash-App-for-Data-Visualization}{
            \textbf{Dash App for Data Visualization}}}% - Data Visualization
    A Dash App written in \textbf{Python3} that visualizes geological data from a CSV file.


\end{minipage}

\end{document}  \documentclass[]{article}

